\chapter{Introduction}

% Networks and Community Detection
Various systems can be modelled as complex networks such as 
social \cite{hagenCrisisCommunicationsAge2018}, citation \cite{mccallumAutomatingConstructionInternet2000, senCollectiveClassificationNetwork2008}, biological \cite{gosakNetworkScienceBiological2018} and transaction \cite{prykeAnalysingConstructionProject2004} networks.
The task of identifying patterns of nodes with common properties, in such networks, is referred to as community detection.
% 
There is an abundant number of community detection methods in literature that approach this problem through modularity optimization \cite{newmanFastAlgorithmDetecting2004, blondelFastUnfoldingCommunities2008, schuetzMultistepGreedyAlgorithm2008}, clique identification \cite{farkasWeightedNetworkModules2007, kumpulaSequentialAlgorithmFast2008}, and spectral optimization \cite{wangCommunityDiscoveryUsing2011, heLaplacianRegularizedGaussian2011}.
% Emergence of graph representation learning
With recent advancements in graph representation learning a new type of methods have emerged which 
utilize context-based learning techniques (e.g., DeepWalk \cite{perozziDeepWalkOnlineLearning2014}, LINE \cite{tangLINELargescaleInformation2015} or Node2Vec \cite{groverNode2vecScalableFeature2016}) to obtain topology-aware node embeddings.
These embeddings are either combined with existing clustering methods \cite{caoIncorporatingNetworkStructure2018, xueCrossdomainNetworkRepresentations2019}
or are jointly optimized with found clusters 
\cite{cavallariLearningCommunityEmbedding2017, rozemberczkiGEMSECGraphEmbedding2019, jiaCommunityGANCommunityDetection2019}
to obtain communities.

% Problem 1: Overlooked features
In the above studies, the dynamic and multimodal characteristics of real-world networks are overlooked. 
These characteristics can manifest as meta-topological features (node and relation types) \cite{caoKnowledgePreservingIncrementalSocial2021}, temporal features, and contentual features (e.g., text and image attributes).
%
Introduction of multimodality contrasts \textit{homophily} assumed by previous methods as \textit{heterophily} and can play an essential role in detecting communities in multimodal networks, as connected nodes may belong to different communities when multiple feature types are considered \cite{zhuHomophilyGraphNeural}.  
% 
While it is common for causal links to be present between these features, it cannot be assumed without extensive domain knowledge.
% 
Various algorithms have been devised to address the issue of temporality and multimodality \cite{greeneTrackingEvolutionCommunities2010, luoDetectingCommunitiesHeterogeneous2021, liCommunityDetectionAttributed2018, faniUserCommunityDetection2020}, 
though as far as we are aware none of the methods are able to address the lossless setting where all the features are incorporated.

% Problem 2: Information variance / asymmetry
Another challenge is information variance present in heterogeneous real-world networks. 
Different node or relation types may have different feature subsets and/or dimensionality.
% Example
Let us consider the Twitter dataset (SDS) we use as a case study in \cref{sec:case_study}. 
This network consists of users, tweets, hashtags, and various relations in-between. 
Here tweets have content as textual features and post dates as temporal features, while users only have biography as textual features, and hashtags have neither. 
Similarly, users form a directed follower relation link, while multiple relations may be present between tweets such as retweet, mention or quote.
The meta-topological information describes important semantics of this network, while varying features and topology can be used to identify individual nodes.
If these characteristics are ignored by a model, then the quality of the communities discovered can be affected. 

% Problem 3: Scale / Topologically incomplete networks
With the emergence of web-scale network datasets (often exceeding billions of nodes), recent advancements have pushed for scalability in graph representation learning \cite{hamiltonInductiveRepresentationLearning2017, yingGraphConvolutionalNeural2018}. 
To this end, graph convolution methods have allowed for inductive inference on unseen nodes no longer requiring storing full graph Laplacian or node embeddings in memory. 
Utilizing this representation function learning helps solve scaling issues faced by many auto-encoder-based and shallow embedding community detection methods \cite{mehtaStochasticBlockmodelsMeet2019, panAdversariallyRegularizedGraph2018, wangDynamicCommunityDetection2017a}.

% TODO: possibly create a list of all the challenges. An overview

In this paper, we propose a novel community detection framework (MGTCOM) that is able to address the aforementioned challenges.
MGTCOM discovers dynamic communities through multimodal feature learning and unsupervised learning with a new sampling technique. 
In particular, our key contributions include:
%
\begin{itemize}[leftmargin=*]
    \item[(i)] A robust method for unsupervised representation learning on multimodal networks
    \item[(ii)] A new sampling technique for unsupervised learning of temporal embeddings
    \item[(iii)] An end-to-end framework optimizing network embeddings, communities, and number of communities in tandem
    \item[(iv)] Extensive evaluation on the quality of various features in multimodal networks
    \item[(v)] Implementation of various graph sampling algorithms found in the literature\footnote{https://github.com/EgorDm/tch-geometric} (See \href{https://anaconda.org/egordm/tch_geometric}{repository}).
\end{itemize}
%
We compare MGTCOM with state-of-the-art methods and demonstrate its robustness on inference tasks.

The rest of the thesis is organized as follows. 
Related works and relevant material is discussed in \cref{sec:related_work} and \cref{sec:preliminaries} respectively.
\cref{sec:approach} covers the details of our frameworks.
In \cref{sec:experiments} we present extensive experimental results including comparison with baseline methods.
\cref{sec:ablation} provides ablation studies to support our design decisions.
Finally, in \cref{sec:case_study} we provide a deep dive into results produced on the Social Distancing Students dataset as a case study.
Source code for the MGTCOM framework can be found on \hyperlink{https://github.com/EgorDm/MGTCOM}{github}\footnote{https://github.com/EgorDm/MGTCOM}.